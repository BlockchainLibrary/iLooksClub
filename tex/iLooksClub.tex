%%%%%%%%%%%%%%%%%%%%%%%%%%%%%%%%%%%%%%%%%
% Journal Article
% LaTeX Template
% Version 1.4 (15/5/16)
%
% This template has been downloaded from:
% http://www.LaTeXTemplates.com
%
% Original author:
% Frits Wenneker (http://www.howtotex.com) with extensive modifications by
% Vel (vel@LaTeXTemplates.com)
%
% License:
% CC BY-NC-SA 3.0 (http://creativecommons.org/licenses/by-nc-sa/3.0/)
%
%%%%%%%%%%%%%%%%%%%%%%%%%%%%%%%%%%%%%%%%%

%----------------------------------------------------------------------------------------
%	PACKAGES AND OTHER DOCUMENT CONFIGURATIONS
%----------------------------------------------------------------------------------------

\documentclass[twoside,twocolumn]{article}

\usepackage{blindtext} % Package to generate dummy text throughout this template 
%\usepackage[utf8]{inputenc} % Package for unicode characters
\usepackage[utf8]{inputenc}
\usepackage{amssymb}
\usepackage{newunicodechar}
\newunicodechar{Ɖ}{\DH}

\usepackage[sc]{mathpazo} % Use the Palatino font
\usepackage[T1]{fontenc} % Use 8-bit encoding that has 256 glyphs
\linespread{1.05} % Line spacing - Palatino needs more space between lines
\usepackage{microtype} % Slightly tweak font spacing for aesthetics

\usepackage[english]{babel} % Language hyphenation and typographical rules

\usepackage[hmarginratio=1:1,top=32mm,columnsep=20pt]{geometry} % Document margins
\usepackage[hang, small,labelfont=bf,up,textfont=it,up]{caption} % Custom captions under/above floats in tables or figures
\usepackage{booktabs} % Horizontal rules in tables

\usepackage{lettrine} % The lettrine is the first enlarged letter at the beginning of the text
\usepackage{url} %added by John Domingue
\usepackage{graphicx} % added by John Domingue
\graphicspath{ {tex/images/} }
\usepackage{enumitem} % Customized lists
\setlist[itemize]{noitemsep} % Make itemize lists more compact

\usepackage{abstract} % Allows abstract customization
\renewcommand{\abstractnamefont}{\normalfont\bfseries} % Set the "Abstract" text to bold
\renewcommand{\abstracttextfont}{\normalfont\small\itshape} % Set the abstract itself to small italic text

\usepackage{titlesec} % Allows customization of titles
\renewcommand\thesection{\Roman{section}} % Roman numerals for the sections
\renewcommand\thesubsection{\roman{subsection}} % roman numerals for subsections
\titleformat{\section}[block]{\large\scshape\centering}{\thesection.}{1em}{} % Change the look of the section titles
\titleformat{\subsection}[block]{\large}{\thesubsection.}{1em}{} % Change the look of the section titles

\usepackage{fancyhdr} % Headers and footers
\pagestyle{fancy} % All pages have headers and footers
\fancyhead{} % Blank out the default header
\fancyfoot{} % Blank out the default footer
\fancyhead[C]{Ethereum Classic Library $\bullet$ October 2016 $\bullet$ Vol. I, No. 1} % Custom header text
\fancyfoot[RO,LE]{\thepage} % Custom footer text

\usepackage{titling} % Customizing the title section

\usepackage[pagebackref]{hyperref} % For hyperlinks in the PDF

%----------------------------------------------------------------------------------------
%	TITLE SECTION
%----------------------------------------------------------------------------------------
\setlength{\droptitle}{-4\baselineskip} % Move the title up

\pretitle{\begin{center}\Huge\bfseries} % Article title formatting
\posttitle{\end{center}} % Article title closing formatting
\title{iLooks club - a revolutionary image consultant app} % Article title
\author{%
\textsc{Prophet Daniel}\thanks{The author would like to thank the Ethereum Classic community.} \\[1ex] % Your name
\normalsize University of Nicosia \\ % Your institution
\normalsize \href{mailto:prophetdaniel@ethereumclassic.org}{prophetdaniel@ethereumclassic.org} % Your email address
% \and % Uncomment if 2 authors are required, duplicate these 4 lines if more
% \textsc{John Domingue}\thanks{} \\[1ex] % Second author's
% \normalsize Knowledge Media Institute, Open University \\ % Second author's institution
% \normalsize \href{mailto:John.domingue@open.ac.uk}{John.domingue@open.ac.uk} % Second
% author's email address
}
\date{\today} % Leave empty to omit a date
\renewcommand{\maketitlehookd}{%
\begin{abstract}
\noindent \Blindtext  
\end{abstract}
}

%----------------------------------------------------------------------------------------

\begin{document}

% Print the title
\maketitle

%----------------------------------------------------------------------------------------
%	ARTICLE CONTENTS
%----------------------------------------------------------------------------------------

\section{Introduction}

% \begin{itemize}
% 	\item What blockchains are and why they are important
% 	\item Although there is a lot of available information on technical aspects
% 	of blockchains there is no coherent space for on blockchain applications
% 	\item Although there is a lot of available information on technical aspects
% 	of blockchains there is no coherent space for on blockchain applications
% 	\item Need to bring the developer, research and policy
% 	communities together (may also others)
% 	\item Proposal of this paper
% \end{itemize}

% \lettrine[nindent=0em,lines=3]{A} decentralized autonomous organization (DAO) is an organization that is run through rules encoded as computer programs called smart contracts and its financial transaction record and program rules are maintained on a blockchain.
% The most famous DAO up to date has as purpose venture capital funding and is called The DAO, which was launched with US\$150 million in crowdfunding in May 2016 and was hacked and drained of approximately US\$50 million in cryptocurrency three weeks later \cite{Price2016}.\par 
% On May 26th of 2016, a paper\cite{Popper2016} first pointed out system vulnerabilities in the operation of The DAO, and recommended a temporary moratorium until all security breaches were fixed. Since its publication and the hack on June 17th, other system vulnerabilities were also found. The hacker had 22 days to study these points, collect all possibilities and take action.\par
% Security breaches are often found by specialists\cite{Perlroth2014}, then publicized as an alerting mechanism to avoid making harm to people. After that happens, hackers are motivated by the amount of money at stake to take action, and they often do way before the fix is deployed.\par
% The DAO was not engineered smart enough to deal with this specific problem. Actually it was not smart in the sense of smartness as we know it. Contrary to what the name suggests, it is also not autonomous yet, because whenever a smart decision needs to be taken, The DAO relies on a voting mechanism to reach its ultimate goal, where the more tokens the voter holds, the more voting power is given to it. In other words, The DAO utilizes human collective intelligence to decide therefore it should be called DO rather than DAO.

\lettrine[nindent=0em,lines=3]{P}eople are so obsessed with their routine these
days that the joy of planning and assembling their overall look seems to be
forgotten in the past. What if the technology could really bring that magical
contentment feelings back to their lives in a more intelligent way, wouldn't it
be amazing? Everybody love buying clothes online, but there is a problem: people
can never get a look they've seen in magazine photos because in reality most
clothes don't fit user's body!

I bought for Christmas 20 gifts and none fitted as they should!
Rose Lee

Because of the disappointing situation people often go to physical stores for
shopping for their selves, where they can try on clothes to make sure they will
fit enough. A more difficult situation happens when they actually intend to buy
clothes to other people like friends or relatives. Or they have a clue of the
user's biometric information or they quit the idea due to the lack of that
important information.

When the individual goes out to physical stores its body information is
inherently carried on and can be used to meet fitting requirements trying the
garments on, looking at the mirror or even asking for fitting advices.

A solution is proposed here to deal with the above problem extending the
application to a complete image consultant application in a way to ease and make
people's lives happier. Wearables and most important wearable gifts should never
mismatch user's body again!


\cite{kurzweil2000age}

%------------------------------------------------
\section{The Technology}

ILooks Club is a fashion app that organizes all the related tasks for an
individual's look identity in ways never thought before, from look conception,
sourcing, scheduling to interacting within social networks. As looks are tightly
related to body shape, the app creates user's 3D body model allowing it to
import any garment it wants to its own realistic or enhanced model.

\subsection{Main features}

\subsubsection{Biometric evolution}

The app records not only the evolution of user's biometric information over time
to but also the main triggers for body change, for example:

Started gym on November 21th.

Even if the user doesn't have access to a 3D full body scanner, biometric
information can be input to the app with simple metric tape measurements.


\subsubsection{Hassle free purchases}

The application is tailored to bring joy and confidence to the purchasing,
delivery and reimbursement processes in such a way to promote the growth of a
loyal community. For that to come true, app user's dreams are fed based on body
looks and feel it is looking for.

For reaching tempting price tags, a search engine is intuitively built-in the
application for obtaining best deals for desired looks. The user is able to
watch and interact with its own model dressed with the desired look where the
total price for achieving the desired look is displayed depending on which
garments the user already has.


\subsubsection{Look scheduler}

The look scheduler allows the looks of the week to be planned in advance. It
auto adjusts according to weather forecast information, combined with laundry
and closet provided information if available.

Don't forget your umbrella today!
Let's switch tomorrow's look, that T-shirt was not washed yet!
Let's switch tomorrow's look, that T-shirt is not in the closet yet!


\subsubsection{Similar biometrics social network}

Similar biometric people are easily discovered all over the world. The app uses
user's friends biometric data to show size matches in its network, but also
suggests to exchange, sell or donate clothes the user doesn't use that often.
Whenever a picture or video media is published, iLooks club API is able to
attach the look information to that media. Once that is done, it becomes
possible to browse the look information of each person present in the picture if
they have consented to make this info public.

This technology allows people to import or even buy exactly the same look the
actor was utilizing in the movie with no questions asked, all the needed
information is already available in the application.


\subsubsection{Advertising opportunity for all}

Because the user's looks are recorded on the blockchain, the property of that
information is exclusively owned by the user itself. That means users have the
option to monetize selling or publicizing that information.
An interesting use case is a celebrity utilizing the look scheduler for its
week. As the looks are thought in advance, pictures or videos posted to social
networks that day are entitled to have the look metadata attached to them.
That's where sponsors are quite interested to make propaganda of their brands
being utilized and automatically published on celebrities social networks.

\subsection{Other features}

\begin{itemize}
	\item Multi-platform complete application (PC, tablets, smartphones, TVs
	\item Although there is a lot of available information on technical aspects
	of blockchains there is no coherent space for on blockchain applications
	\item Although there is a lot of available information on technical aspects
	of blockchains there is no coherent space for on blockchain applications
	\item Need to bring the developer, research and policy
	communities together (may also others)
	\item Proposal of this paper
	\item Multi-platform complete application (PC, tablets, smartphones, TVs).
	\item Intelligent fashion advisor. Selection by fashion trends, fashion experts, and desired activity (gym, sleep, work, appointment).
	\item Psychological tips for reaching desired look based on biometric patterns matching consumer biometric model. �Try using vertical stripes for looking slimmer!� User selects the psychological engine depending on desired traits.
	\item Look suggestions based on likes and dislikes user provides.
	\item 3D visualization of realistic consumer biometric model and enhanced biometric model (more elegant).
	\item Clothes exchange club (social network) with similar biometric measures people.
	\item Share my look with selected list of friends. Import app user friend's look to its biometric model or its enhanced model.
	\item Displaying biometric models of famous people wearing their fashion looks with even price info. Simulation of same look with app user's biometric model or app user's enhanced model.
	\item Look overall thermal resistance calculation. Tells for example how many calories per hour are going to be spent at selected ambient temperature and sun exposition.
	\item Donation campaigns score system whenever app user donates, discount bonuses are added up to a given limit.
	\item Donation suggestions based on clothes not utilized in the closet.
	``Donating those 4 garments you don't actually use, that garment in your wish list will cost to you \$ xx.xx''.
\end{itemize}


\section{Conclusion}

I proposed a novel way to manage an individual's look identity able to trigger
contentment feelings easing and making its life more fun through interaction
with similar biometric people otherwise unknown to each other. This feature is
able to promote the creation of an interesting social network.
Users with enough followers are able to make profit just by utilizing sponsor's
brands and allowing the look information to be publicized.
The access to user's biometric evolution can help health services to better
understand if user's habits are really being positive.
Manufacturing is affected in positive ways since whenever a look order is
placed, the user's biometric information is sent to them to make sure garments
will perfectly fit reducing the return rates and eliminating necessary
adjustments due to unfit situations.
Donation campaigns within similar biometric people are able to reduce
inequality, promote inclusion of otherwise marginalized people toward a more
sustainable world.
All these aspects combined are able to leverage this project into a true social
transformation agent.

%Maecenas sed ultricies felis. Sed imperdiet dictum arcu a egestas. 
%\begin{itemize}
%\item Donec dolor arcu, rutrum id molestie in, viverra sed diam
%\item Curabitur feugiat
%\item turpis sed auctor facilisis
%\item arcu eros accumsan lorem, at posuere mi diam sit amet tortor
%\item Fusce fermentum, mi sit amet euismod rutrum
%\item sem lorem molestie diam, iaculis aliquet sapien tortor non nisi
%\item Pellentesque bibendum pretium aliquet
%\end{itemize}
%\blindtext % Dummy text

%Text requiring further explanation\footnote{Example footnote}.

%------------------------------------------------

%\section{Results}

%\begin{table}
%\caption{Example table}
%\centering
%\begin{tabular}{llr}
%\toprule
%\multicolumn{2}{c}{Name} \\
%\cmidrule(r){1-2}
%First name & Last Name & Grade \\
%\midrule
%John & Doe & $7.5$ \\
%Richard & Miles & $2$ \\
%\bottomrule
%\end{tabular}
%\end{table}

%\blindtext % Dummy text

%\begin{equation}
%\label{eq:emc}
%e = mc^2
%\end{equation}

%\blindtext % Dummy text

%------------------------------------------------

%\section{Discussion}

%\subsection{Subsection One}

%A statement requiring citation \cite{Figueredo2009}.
%\blindtext % Dummy text

%\subsection{Subsection Two}

%\blindtext % Dummy text

%----------------------------------------------------------------------------------------
%	REFERENCE LIST
%----------------------------------------------------------------------------------------

\bibliographystyle{abbrv}  
\bibliography{tex/bibliography}

%----------------------------------------------------------------------------------------

\end{document}
